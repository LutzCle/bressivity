\pgfmathsetmacro{\cardwidth}{6.35}%{6.35}%{5.715} %
%% width of card in cm. 5.715 = 2.25 in, 6.35 = 2.5 in
\pgfmathsetmacro{\cardheight}{8.875}
%% height of a card in cm. 8.875 = 3.5 in

\pgfmathsetmacro{\borderwidth}{0.3}%{0.0167*\cardwidth}%{0.1}
%% white border around the card

\pgfmathsetmacro{\imagewidth}{1*\cardwidth-2*\borderwidth}
\pgfmathsetmacro{\imageheight}{1*\cardheight-2*\borderwidth}
%% the image is AUTOMATICALLY RESIZED to fit the \imagewidth * \imageheight zone
%% ratio for poker-sized deck:
%%  w: 63.5-2*3 = 57.5
%%    in px : 680
%%  h: 88.9-2*3 = 82.9
%%    in px : 980 (300 dpi)
%%  w/h = 0.6936
%%  h/w = 1.4417 (~13/9, ie 3R photo format)

\pgfmathsetmacro{\stripwidth}{0.100*\cardwidth}%{0.7} % 0.12
%% strip is background for indices and title
\pgfmathsetmacro{\strippadding}{0.0333*\cardwidth}%{0.2}
%% padding around strips
\pgfmathsetmacro{\textpadding}{0.0167*\cardwidth}%{0.1}
%% padding for text

\pgfmathsetmacro{\ratiocontent}{0.66}
%% % of vertical space used for content

\def\shapeCard{(0,0) rectangle (\cardwidth,\cardheight)}
%% main card (with rounded corners)
\def\shapeBackground{(\borderwidth,\borderwidth) rectangle (\cardwidth-\borderwidth,\cardheight-\borderwidth)}
%% main rectangle of cards, where the image is printed

\pgfmathsetmacro{\titley}{\strippadding+1*\stripwidth+\borderwidth}
%% vertical position of title : center of area
\pgfmathsetmacro{\titlex}{\stripwidth + 2*\strippadding + \borderwidth + \textpadding + 0.5*(\cardwidth-\stripwidth-3*\strippadding-2*\borderwidth-2*\textpadding)}
%% vertical position of title : center of area

\def\shapeTitle{(2*\strippadding+\stripwidth+\borderwidth,\strippadding+\borderwidth) rectangle (\cardwidth-\strippadding-\borderwidth,2*\stripwidth+\strippadding+\borderwidth)}
%% backgound zone for title

%% title (front)/date (back) : upper left to  lower right
\def\shapeContentArea{(2*\strippadding+\stripwidth+\borderwidth,\cardheight) rectangle (\cardwidth,\cardheight-\borderwidth-\ratiocontent*\cardheight)}
%% main text (back of the card), relative to cardheight and cardwidth to "cut" top and right rounded corners

\def\shapeStrip{(\strippadding+\borderwidth,\cardheight) rectangle (\strippadding+\stripwidth+\borderwidth,\strippadding+\borderwidth)}
%% vertical left long text

%% tickzset styles
\tikzset{
  cardcorners/.style={
    inner sep=0,
    outer sep=0,
    rounded corners=\strippadding *1cm %
    %% rounded corners of cards
  },
  elementcorners/.style={
    rounded corners= \textpadding *1cm % 0.1cm
    %% rounded corners of strips in the card
  },
  stripshadow/.style={
    drop shadow={
        opacity=.2,
        color=black
    }
    %% shadow for strips
  },
  strip/.style={
    elementcorners,
    %% stripshadow
    %% main style for strips
  },
  cardimage/.style={
    path picture={
      \node at (0.5*\cardwidth,0.5*\cardheight) {
          \includegraphics[height=\imageheight cm, width=\imagewidth cm]{#1}
      };
    }
    %% main picture
  },
}

\newcommand{\carddebug}{
  \draw [step=0.25in,help lines] (0,0) grid (\cardwidth,\cardheight);
}
%% debug

\newcommand{\cardborder}{
  \draw[lightgray, cardcorners]  \shapeCard; %[lightgray, cardcorners] 
}
%% main card, with border

\newcommand{\cardbackground}[1]{
  \draw[lightgray,cardimage=#1] \shapeBackground;
}
%% main zone, where the image is printed


%% Left strip of the card, with bg color (arg #1) and text (arg #2) written vertically
\newcommand{\cardtype}[2]{
  % The \clip command does not allow options, therefore  we have to use a scope to set the even odd rule.
  \begin{scope}[even odd rule]
    % Define a clipping path. All paths outside shapeCard will be cut because the even odd rule is set.
    \clip[cardcorners] \shapeBackground;
    % Since the even odd rule is set, only the card will be filled.
    \fill[strip,#1] \shapeStrip (\strippadding+0.5*\stripwidth+\borderwidth,\cardheight-\borderwidth-\textpadding) node[rotate=90, left] {
      \color{white}\uppercase{#2}%
    };
  \end{scope}
}

%% Title (arg #1) of the card, written on shapeTitle
\newcommand{\cardtitle}[1]{
  \fill[elementcorners,titlebg,opacity=.85] \shapeTitle;
  \node[align=center, text width=(\cardwidth-\stripwidth-2*\strippadding-2*\borderwidth-2*\textpadding)*1cm] at (\titlex,\titley) {%
    \color{white}\uppercase{#1}%
  };
}

%% Maincontent, written at the back: date (arg #1) and description (arg #2)
\newcommand{\cardcontent}[2]{
  \begin{scope}[even odd rule]
    \clip \shapeBackground;
    \fill[elementcorners,contentbg,opacity=.85] \shapeContentArea;
  \end{scope}
  \node[below right, align=justify, text width=(\imagewidth-2*\strippadding-\stripwidth-2*\textpadding)*1cm] at (2*\strippadding+\stripwidth+\borderwidth,\cardheight-\strippadding-\borderwidth) {
    \centerline{\bf\Large #1}\\%
    {\footnotesize #2}
  };
}

%% Print vertical credits (arg #2) on a side (front). Text color (arg #1) is white by default
%% -0.105 shift so that the text is aligned along the border
\newcommand{\cardcredits}[2][white]{
  \node[rotate=90, left] at (\cardwidth-\borderwidth-0.105,\cardheight-\borderwidth) {
    {\color{#1}\tiny #2}
  };
}

%% given card types, shorthand
\newcommand{\cardtypePop}{\cardtype{popbg}{{Culture populaire}}}
\newcommand{\cardtypeHard}{\cardtype{hardbg}{{Matériel}}}
\newcommand{\cardtypeSoft}{\cardtype{softbg}{{Logiciel}}}
\newcommand{\cardtypeLang}{\cardtype{langbg}{{Langages de programmation}}}
\newcommand{\cardtypeTheory}{\cardtype{theorybg}{{Fondements et concepts}}}
\newcommand{\cardtypeGame}{\cardtype{gamebg}{{Jeux vidéos}}}
\newcommand{\cardtypeWeb}{\cardtype{webbg}{{Réseaux et web}}}
\newcommand{\cardtypeMan}{\cardtype{manbg}{{Personnalité}}}

